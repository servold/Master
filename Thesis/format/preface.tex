\setcounter{page}{3}
\chapter*{Abstract}

The clustering problem is one of the fundamental problem in unsupervised machine learning, and arises in a wide scope of applications. The clustering problem is a nonconvex and nonsmooth optimization problem. We propose two clustering center-based algorithms, each tackles a different distance function. We prove the global convergence of these algorithm to a critical point via a new methodology which in based on the powerful Kurdyka-{\L}ojasiewicz property. As an illustration of the results, we present numerical tests which demonstrate the effectiveness of proposed algorithms.


\addcontentsline{toc}{chapter}{Abstract}

\chapter*{Acknowledgements}

I would like to thank my advisor, Prof. Marc Teboulle, for introducing me to the interesting world of continuous optimization during my graduate studies. I am deeply grateful to Prof. Teboulle for motivating and encouraging me to research the clustering problem and for his helpful ideas how to tackle this interesting and important problem.

I would like to acknowledge my co-advisor, Prof. Shoham Sabach, for his insightful comments and constructive criticisms at different stages of my research and for his dedication and patience in helping me to complete this thesis.

Most importantly, none of this would have been possible without the love and patience of my family, to whom this thesis is dedicated to. My family has been a constant source of love, concern, support and strength.

Finally, I appreciate the financial support from Tel-Aviv university which enabled me to focus on my research, and achieve this thesis.
\\
\\
\noindent February 2016\\
\noindent Sergey Voldman


\addcontentsline{toc}{chapter}{Acknowledgements}

\nobreak
