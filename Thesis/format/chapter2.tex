 \chapter{Problem Reformulation and Mathematical Tools} \label{Chapter2}

\noindent \noindent \hrulefill

\section{Reformulation of the Clustering Problem} \label{State_Clustering_Reformulation}

We begin with a reformulation of the clustering problem which will be the basis for our developments in this work. The reformulation is based on the following fact:
\begin{equation*}
	\min\limits_{1 \leq l \leq k} u_l = \min \left\lbrace \langle u,v \rangle : v \in \Delta \right\rbrace ,
\end{equation*}
where $\Delta$ denotes the well-known simplex defined by
\begin{equation*}
	\Delta = \left\lbrace u \in \mathbb{R}^k : \sum\limits_{l=1}^{k} u_l = 1, \: u \geq 0 \right\rbrace .
\end{equation*}
Using this fact in Problem (\ref{StateEq1}) and introducing new variables $w^i \in \mathbb{R}^k$, $i=1,2, \ldots, m$, gives a smooth reformulation of the clustering problem
\begin{equation}
	\min\limits_{x \in \mathbb{R}^{nk}} \sum\limits_{i=1}^{m} \min\limits_{w^i \in \Delta} \langle w^i , d^i(x) \rangle , \label{StateEq2}
\end{equation}
where 
\begin{equation*}
d^{i}(x) = (d(x^1,a^i), d(x^2,a^i), \ldots , d(x^k,a^i)) \in \mathbb{R}^k, \quad i=1, 2, \ldots , m.
\end{equation*}
Replacing further the constraint $w^i \in \Delta$ by adding the indicator function $\delta_{\Delta}(\cdot)$, which is defined to be $0$ in $\Delta$ and $\infty$ otherwise, to the objective function, results in a equivalent formulation
\begin{equation}
	\min\limits_{x \in \mathbb{R}^{nk} , w \in \mathbb{R}^{km}} \left\lbrace \sum\limits_{i=1}^{m} \left( \langle w^i , d^i(x) \rangle + \delta_{\Delta}(w^i) \right) \right\rbrace , \label{StateEq3}
\end{equation}
where $w = (w^1, w^2, \ldots , w^m) \in \mathbb{R}^{km}$.
Finally, for the simplicity of the yet to come expositions, we define the following functions
\begin{center}
$H(w,x) := \sum\limits_{i=1}^{m} H^i(w,x) = \sum\limits_{i=1}^{m} \langle w^i , d^i(x) \rangle \quad$ and $\quad G(w) = \sum\limits_{i=1}^{m} G^i(w^i) := \sum\limits_{i=1}^{m} \delta_{\Delta}(w^i) .$
\end{center}

Replacing the terms in Problem (\ref{StateEq3}) with the functions defined above gives a compact equivalent form of the original clustering problem

\begin{equation}
	\min \left\lbrace \Psi(z) := H(w,x) + G(w) \mid z := (w,x) \in \mathbb{R}^{km} \times \mathbb{R}^{nk} \right\rbrace . \label{StateEq4}
\end{equation}

\section{Convergence Methodology} \label{State_PALM_Theory}

In this subsection we give a brief review of the main developments established in \cite
{BST2014}. These developments include on one hand the proximal alternating linearized minimization (PALM) algorithm and on the other hand, a general procedure for proving global convergence of generic algorithm which will play a central rule in this work. First, let us recall several definitions which are needed for the upcoming discussion.

\begin{definition}[Subdifferentials] \label{subdiff_def}
	Let $\sigma : \rr^d \rightarrow (-\infty,+\infty]$ be a proper and lower semicontinuous function.
	\begin{enumerate}[(i)]
		\item For a given $x \in dom\;\sigma := \left\lbrace x \in \rr^d : \; \sigma(x) < \infty \right\rbrace$, the \textit{Fr\'echet subdifferential} of $\sigma$ at $x$, written $\widehat{\partial}\sigma(x)$, is the set of all vectors $u \in \rr^d$ which satisfy
		\begin{equation*}
			\lim\limits_{y \neq x} \inf\limits_{y \rightarrow x} \frac{\sigma(y) - \sigma(x) - \left\langle u, y-x \right\rangle}{\norm{y-x}} \geq 0.
		\end{equation*}
		When $x \notin dom\sigma$, we set $\widehat{\partial}\sigma(x) = \emptyset$.
		\item The \textit{limiting-subdifferential}, or subdifferential in short, of $\sigma$ at $x \in \rr^n$, written $\partial\sigma(x)$, is defined through the following closure process
		\begin{equation*}
			\partial\sigma(x) := \left\lbrace u \in \rr^d : \exists x^k \rightarrow x, \; \sigma(x^k)\rightarrow \sigma(x) \text{ and } u^k \in \widehat{\partial}\sigma(x^k) \text{ as } k \rightarrow \infty \right\rbrace.
		\end{equation*}
	\end{enumerate}
\end{definition}
In the nonsmooth context, as in the smooth case, the well-known Fermat's rule remains unchanged, that is, if $x \in \rr^d$ is a local minimizer of $\sigma$ then $0 \in \partial\sigma(x)$. Points whose subdifferential contains $0$ are called \textit{critical points}, and the set of all critical points of $\sigma$ is denoted by crit$\sigma$.\medskip

Now we present the Kurdyka-{\L}ojasiewicz property, which plays a central role in the general methodology which was developed in \cite{BST2014}. Let $\eta \in (0,+\infty]$. Denote the following class of concave functions
\begin{equation*}
	\Phi_{\eta} = \left\lbrace \varphi \in C\left([0,\eta), \rr_+ \right)  : \; \varphi \in C^1\left((0,\eta)\right), \; \varphi'>0, \; \varphi(0)=0 \right\rbrace .
\end{equation*}

\begin{definition}[Kurdyka-{\L}ojasiewicz property]
	Let $\sigma: \rr^d \rightarrow (-\infty,+\infty]$ be proper and lower semicontinuous.
	\begin{enumerate}[(i)]
		\item The function $\sigma$ is said to have the \textit{Kurdyka-{\L}ojasiewicz (KL) property} at $\overline{u} \in dom \; \partial\sigma :=  \left\lbrace u \in \rr^d : \; \partial\sigma \neq \emptyset \right\rbrace$ if there exist $\eta \in (0,+\infty]$, a neighborhood $U$ of $\overline{u}$ and a function $\varphi \in \Phi_{\eta}$, such that for all
		\begin{equation*}
			u \in U \cap \left\lbrace x \in \rr^d : \; \sigma(\overline{u}) < \sigma(x) < \sigma(\overline{u}) + \eta \right\rbrace,
		\end{equation*}
		the following inequality holds
		\begin{equation*}
			\varphi'(\sigma(u) - \sigma(\overline{u}))dist(0,\partial\sigma(u)) \geq 1,
		\end{equation*}
		where $dist(x,S) := \inf \left\lbrace \norm{y-x} : \; y \in S\right\rbrace$ denotes the distance from $x \in \rr^d$ to $S \subset \rr^d$.
		\item If $\sigma$ satisfy the KL property at each point of $dom\;\sigma$ then $\sigma$ is called a \textit{KL function}.
	\end{enumerate}
\end{definition}

As it can be seen from the definition above, verifying that a given function satisfies the KL property is quite involved. This can be overcome by using an important result of Bolte et al. from 2006 (see \textbf{[give reference]}). Before presenting this result we will recall the definition of semi-algebraic function.

\begin{definition}[Semi-algebraic sets and functions] 
\begin{enumerate}[(i)]
	\item A subset $S \subset \rr^d$ is a real semi-algebraic set if there exists a finite number of real polynomial functions $g_{ij}, h_{ij}: \rr^d \rightarrow \rr$ such that
	\begin{equation*}
		S = \bigcup\limits_{j=1}^{p} \bigcap\limits_{i=1}^{q} \left\lbrace u \in \rr^d : \; g_{ij}=0 \text{ and } h_{ij}(u)<0 \right\rbrace
	\end{equation*}
	\item A function $h:\rr^d \rightarrow (-\infty, +\infty]$ is called semi-algebraic if its graph
	\begin{equation*}
		\left\lbrace (u,t)\in\rr^{d+1} : \; h(u)=t \right\rbrace ,
	\end{equation*}
	is a semi-algebraic subset of $\rr^{d+1}$.
\end{enumerate}
\end{definition}

\begin{theorem} \label{SA_is_KL}
Let $\sigma:\rr^d \rightarrow (-\infty, +\infty]$ be a proper and lower semicontinuous function. If $\sigma$ is semi-algebraic then if satisfies the KL property at any point of dom $\sigma$.
\end{theorem}
The class of semi-algebraic functions is very broad, it includes real polynomial functions, indicator functions of semi-algebraic sets, finite sums and products of semi-algebraic functions, composition of semi-algebraic functions, and many more.\medskip

Attouch et al. \cite{AB2009, ABS2013} established convergence of the sequences generated by the proximal Gauss-Seidel scheme in the general nonconvex and nonsmooth setting, and by the proximal-forward-backward (aka Proximal Gradient) algorithm applied to the nonconvex and nonsmooth minimization of the sum of a nonsmooth function with a smooth one. This approach assumes that the objective function to be minimized satisfies the Kurdyka-{\L}ojasiewicz (KL) property. The convergence results were further extended in the recent work by Bolte et al. \cite{BST2014}, to the PALM algorithm which is a novel algorithm that combines the two basic and old ideas of Alternating Minimization and Proximal Gradient (see more details below). Additional contribution of \cite{BST2014} is the general methodology to prove convergence of generic algorithm in the setting of nonconvex and nonsmooth optimization problems.

Equipped with these definitions, we present the general methodology that will be used several times throughout this work. Let $\Psi : \rr^d \rightarrow ( -\infty, +\infty ]$ be a proper and lower semicontinuous function which is bounded from below and consider the problem
\begin{equation*}
	(P) \quad \min \left\lbrace \Psi(z) : z \in \rr^d \right\rbrace .
\end{equation*} 
Suppose that we are given a generic algorithm $\mathcal{A}$ which generates a sequence $\left\lbrace z^k \right\rbrace_{k \in \mathbb{N}}$ via the following scheme:
\begin{equation*}
	z^0 \in \mathbb{R}^d, \: z^{k+1} \in \mathcal{A}\left(z^k\right), \quad k=0,1,\ldots.
\end{equation*}

The purpose of the proposed methodology is to assure the convergence of the whole sequence $\left\lbrace z^k \right\rbrace_{k \in \mathbb{N}}$ to a critical point of $\Psi$. The set of all limit points is denoted by $\omega\left(z^0\right)$, and defined by
\begin{equation*}
	\left\lbrace \overline{z} \in \rr^d : \exists \text{ an increasing sequence of integers } \left\lbrace k_l\right\rbrace_{l \in \nn} \text{ such that } z^{k_l} \rightarrow \overline{z} \text{ as } l \rightarrow \infty \right\rbrace .
\end{equation*}

\begin{definition}
	Let $\sigma: \rr^d \rightarrow (-\infty, +\infty]$ be a proper and lsc function. A sequence $\left\lbrace z^k \right\rbrace_{k \in \nn}$ is called \textit{a gradient-like descent sequence} for $\sigma$ if for all $k \in \nn$ the following two conditions hold:
	\begin{enumerate}[(C1)]
		\item \textit{Sufficient decrease property:} There exists a positive scalar $\rho_1$ such that
		\begin{equation*}
			\rho_1 \norm{z^{k+1} - z^k}^2 \leq \sigma\left( z^k \right) - \sigma \left( z^{k+1} \right) .
		\end{equation*}
		\item \textit{A subgradient lower bound for the iterates gap:}
		\begin{itemize}
			\item[$-$] $\left\lbrace z^k \right\rbrace_{k \in \nn}$ is bounded.
			\item[$-$] There exists a positive scalar $\rho_2$ such that
			\begin{equation*}
				\norm{w^{k+1}} \leq \rho_2 \norm{z^{k+1} - z^k}, \; w^{k+1} \in \partial\sigma \left( z^{k+1}\right).
			\end{equation*}
		\end{itemize}
	\end{enumerate}
\end{definition}

The two conditions (C1) and (C2) defining a gradient-like descent sequence for a given $\sigma$ are typical for any descent type algorithm, and provide the basic tools to prove that the limit of any convergent subsequence of $\left\lbrace z^k \right\rbrace_{k \in \nn}$ is a critical point of $\sigma$. More precisely, from \cite{BST2014} we have

\begin{lemma}
If $\left\lbrace z^k \right\rbrace_{k \in \nn}$ is a gradient-like descent sequence for a given function $\sigma$, which is lsc and proper on $\rr^d$, then $\omega\left(z^0\right)$ is a nonempty, compact and connected set, and we have
\begin{equation*}
	\lim_{k\rightarrow \infty} dist\left( z^k, \omega\left(z^0\right)\right) = 0.
\end{equation*}
\end{lemma}

This result can thus be applied to any algorithm that produces a gradient-like descent to establish convergence in accumulation points. The main goal is to establish global convergence, i.e., that the whole sequence converges to a critical point of $\sigma$. This can be achieved by imposing an additional assumption on the class of functions $\sigma$, it must satisfy the Kurdyka-{L}ojasiewiez property.\medskip

As proven in \cite{BST2014}, relying on a key uniformization of the KL property it is possible to establish global convergence of any gradient-like descent sequence $\left\lbrace z^k \right\rbrace_{k \in \nn}$, independently of the algorithm used. Verifying the KL property of a given function might often be a difficult task. However, thanks to a \Cref{SA_is_KL}, any proper and lsc function $\sigma$ which is semi-algebraic satisfies the KL property at any point in dom$\sigma$. We summarize the general methodology and convergence results of \cite{BST2014} in the following abstract convergence result.

\begin{theorem}
Let $\sigma:\rr^d \rightarrow (-\infty,\infty]$ be a proper, lsc and semi-algebraic function with $\inf \sigma > -\infty$, and assume that $\left\lbrace z^k \right\rbrace_{k \in \nn}$ is a gradient-like descent sequence for $\sigma$. If $\omega\left( z^0 \right) \subset crit\sigma$ then the sequence $\left\lbrace z^k \right\rbrace_{k \in \nn}$ convergences to a critical point $z^{*}$ of $\sigma$.
\end{theorem}

\begin{remark}
Under the premises of this theorem, it is also possible to derive a rate of convergence result for the sequence $\left\lbrace z^k \right\rbrace_{k \in \nn}$ of the form $\norm{z^k - z^{*}} \leq Ck^{-\gamma}$, for some positive constant $C$ and where $\gamma>0$ is a so-called KL exponent.
\end{remark}

Finally, we present the proximal alternating linearized minimization (PALM) algorithm which solves the nonconvex and nonsmooth minimization problem of the following form
\begin{center}
(\textit{M})\quad minimize $\Psi(x,y):=f(x)+g(y)+H(x,y)$ over all $(x,y) \in \mathbb{R}^n \times \mathbb{R}^m$,
\end{center}
where $f:\mathbb{R}^n \rightarrow \left(-\infty,+\infty\right]$ and $g:\mathbb{R}^n \rightarrow \left(-\infty,+\infty\right]$ are proper and lower semicontinuous functions while $H:\mathbb{R}^n \times \mathbb{R}^m \rightarrow \mathbb{R}$ is a $C^1$ function. In addition, partial gradients of $H$ are Lipschitz continuous, namely, $H(\cdot, y) \in C^{1,1}_{L_1(y)}$ and $H(x,\cdot) \in C^{1,1}_{L_2(x)}$.\medskip

As mentioned in \cite{BST2014} the PALM algorithm is nothing but alternating the classical proximal gradient over the two blocks $(x,y)$. This leads towards the following approximations 
\begin{equation*}
	\widehat{\Psi}\left(x,y^k\right) = \left\langle x-x^k, \nabla_x H\left( x^k, y^k \right)\right\rangle + \frac{c_k}{2}\norm{x-x^k}^2 + f(x), \; (c_k>0),
\end{equation*}
and
\begin{equation*}
	\widetilde{\Psi}\left(x^{k+1},y\right) = \left\langle y-y^k, \nabla_y H\left( x^{k+1}, y^k \right)\right\rangle + \frac{d_k}{2}\norm{y-y^k}^2 + g(y), \; (d_k>0).
\end{equation*}
Thus, PALM can be summarized as follows
\begin{center}
$x^{k+1} \in \arg\!\min \left\lbrace \widehat{\Psi}(x,y^k) : \; x \in \rr^n \right\rbrace$ \quad and \quad $y^{k+1} \in \arg\!\min \left\lbrace \widetilde{\Psi}\left(x^{k+1},y\right) : \; y \in \rr^m \right\rbrace$.
\end{center}
Assuming $\Psi$ is KL function and the generated sequence by PALM, $\left\lbrace \left( x^k, y^k\right)\right\rbrace_{k \in \nn}$, is bounded, Bolte et al. \cite{BST2014} proved that the sequence is a gradient-like descent sequence, and thus it converges to a critical point of $\Psi$. 
