 \chapter{Returning to k-means} \label{Chapter5}

\noindent \noindent \hrulefill

\section{Similarity of KPALM to k-means}

The famous k-means algorithm has close relation to KPALM algorithm. k-means alternates between cluster assignment and centers update steps as well. In detail, we can write its steps in the following manner

\begin{framed}
\noindent \textbf{k-means}
\begin{enumerate}[(1)]
	\item Initialization: $x(0) \in \mathbb{R}^{nk}$.
	\item General step $\left( t=0,1, \ldots \right)$:
	\begin{enumerate}[(2.1)]
		\item Cluster assignment: for $i=1, 2, \ldots ,m$ compute
		\begin{equation}
			w^i(t+1) = \arg\!\min\limits_{w^i \in \Delta} \left\lbrace \langle w^i , d^i(x(t)) \rangle\right\rbrace . \label{StateEq12}
		\end{equation}
		\item Center update: for $l=1, 2, \ldots ,k$ compute
		\begin{equation}
			x^l(t+1) = \frac{\sum_{i=1}^{m} w^i_l(t+1) a^i}{\sum_{i=1}^{m} w^i_l(t+1)} . \label{StateEq13}
		\end{equation}
%		\item Stopping criteria: halt if 
%		\begin{equation}
%			\forall 1 \leq l \leq k \quad C^l(t+1)=C^l(t) \label{StateEq15}
%		\end{equation}
	\end{enumerate}
\end{enumerate}
\end{framed}

It is easy to see that if we take $\alpha_i(t) = 0$ for all $1 \leq i \leq m$ and $t \in \mathbb{N}$, then KPALM becomes k-means. We aim to use the theory described in \Cref{State_PALM_Theory} once again and show that the sequence generated by k-means converges to a critical point of $\Psi(\cdot)$, as defined is (\ref{StateEq4}). The sufficient decrease proof of \Cref{State_Clustering_SqNorm} collapses in this case, since it is based on Assumption \ref{StateMainAssum}(\ref{StateMainAssum1}), that is, $\alpha_i(t) > \underline{\alpha_i} > 0$, for all $t \in \mathbb{N}$ and $i=1,2, \ldots, m$. However, the proof of the subgradient lower bound for the iterates gap property follows through as is. In the following discussion we present the means to treat the case that $\alpha_i(t) = 0$, and prove the sufficient decrease property.

\begin{lemma} \label{StateLemma_x_bounds_w}
Let $\left\lbrace z(t) \right\rbrace_{t \in \mathbb{N}}$ be the sequence generated by k-means. Then, there exists $c > 0$ such that
\begin{equation*}
	\|w^i(t+1)-w^i(t)\| \leq c \|x(t+1)-x(t)\| , \quad \forall \: i=1,2, \ldots\ m, \: t \in \mathbb{N} .
\end{equation*}
\end{lemma}

\begin{proof}
At each iteration k-means partitions the set $\mathcal{A}$ into $k$ clusters, and the center of each cluster is its mean. Since the number of these partitions if finite, there exists a finite set $\mathcal{C} = \left\lbrace x^1,x^2, \ldots, x^N \right\rbrace\\ \subset \mathbb{R}^{nk}$ such that for all $t \in \mathbb{N}$, $x(t) \in \mathcal{C}$. We denote
\begin{equation*}
	r = \min\limits_{1 \leq j < l \leq N} \|x^j-x^l\|,
\end{equation*}
and set $c = \sqrt{2}/r$.
At each iteration, the point $a^i$ can move from one cluster to another, hence
\begin{equation*}
	\|w^i(t+1)-w^i(t)\| \leq \sqrt{2} .
\end{equation*}
Therefore, combining these arguments yields
\begin{equation*}
	\frac{\|w^i(t+1-w^i(t)\|}{\|x(t+1)-x(t)\|} \leq \frac{\sqrt{2}}{r} .
\end{equation*}
In case that $x(t+1)=x(t)$, this implies that none of the clusters has changed, hence we proved the statement in both cases.
\end{proof}

Equipped with the last lemma we briefly prove the sufficient decrease property of k-means.

\begin{proposition}[Sufficient decrease property for k-means sequence]
Let $\left\lbrace z(t) \right\rbrace_{t \in \mathbb{N}}$ be the sequence generated by k-means. Then, there exists $\rho_1 > 0$ such that 
\begin{equation*}
	\rho_1 \|z(t+1) - z(t)\|^2 \leq \Psi_{\varepsilon}(z(t)) - \Psi_{\varepsilon}(z(t+1)) \quad \forall \: t \in \mathbb{N} .
\end{equation*}
\end{proposition}

\begin{proof}
The function $x \mapsto H(w(t),x)$ remains strongly convex with parameter $\beta(w(t))$ (see (\ref{StateEq17})), hence we have a sufficient decrease in the $x$ variable, namely,
\begin{equation}
	\frac{\underline{\beta}}{2} \|x(t+1)-x(t)\|^2 \leq H(w(t),x(t)) - H(w(t+1),x(t+1)) . \label{StateEq70}
\end{equation}
Setting $\rho_1 = \underline{\beta} /2(1 + mc^2)$, we can write
\begin{align*}
	\rho_1 \|z(t+1)-z(t)\|^2 &= \rho_1 \sum\limits_{i=1}^{m} \|w^i(t+1)-w^i(t)\|^2 + \rho_1 \|x(t+1)-x(t)\|^2 \\
	&\leq \rho_1 (1 + mc^2) \|x(t+1) - x(t)\|^2 \\
	&\leq H(w(t),x(t)) - H(w(t+1),x(t+1)) \\
	&= \Psi(z(t)) - \Psi(z(t+1))
\end{align*}
where the first inequality follows from \Cref{StateLemma_x_bounds_w}, the second follows from (\ref{StateEq70}), and the last equality follows from the fact that $G(w(t))=0$, for all $t \in \mathbb{N}$.
\end{proof}

\section{k-means Local Minima Convergence Proof}

In this section we present a simple and direct proof that k-means converges to local minima. We start with rewriting the k-means algorithm, in its most familiar form
\begin{framed}
\noindent \textbf{k-means}
\begin{enumerate}[(1)]
	\item Initialization: $x(0) \in \mathbb{R}^{nk}$.
	\item General step $\left( t=0,1, \ldots \right)$:
	\begin{enumerate}[(2.1)]
		\item Cluster assignment: for $i=1, 2, \ldots ,m$ compute
		\begin{equation}
			C^l(t+1) = \left\lbrace a \in \mathcal{A} \mid \| a - x^l(t) \| \leq \|a - x^j(t) \|, \quad \forall 1 \leq l \leq k \right\rbrace . \label{StateEq20}
		\end{equation}
		\item Center update: for $l=1, 2, \ldots ,k$ compute
		\begin{equation}
			x^l(t+1) = mean(C^l(t+1)) := \frac{1}{\left| C^l(t+1) \right|} \sum\limits_{a \in C^l(t+1)} a . \label{StateEq21}
		\end{equation}
		\item Stopping criteria: halt if 
		\begin{equation}
			\forall 1 \leq l \leq k \quad C^l(t+1)=C^l(t) \label{StateEq22}
		\end{equation}
	\end{enumerate}
\end{enumerate}
\end{framed}

As in KPALM, k-means needs Assumption \ref{StateMainAssum}(\ref{StateMainAssum2}) for step (\ref{StateEq21}) to be well defined. In order to prove the convergence of k-means to local minimum, we will need to following assumption.

\begin{assumption} \label{StateEq23}
Let $t \in \mathbb{N}$ be the final iteration of k-means run, then we assume that each $a \in \mathcal{A}$ belongs exclusively to single cluster $C^l(t)$.
\end{assumption}

For any $x \in \mathbb{R}^{nk}$ we denote the super-partition of $\mathcal{A}$ with respect to $x$ by $\overline{C^l}(x) = \left\lbrace a \in \mathcal{A} \mid \right. \left. \|a - x^l\| \leq \|a - x^j\| , \quad \forall j \neq l \right\rbrace$, for all $1 \leq l \leq k$, and the sub-partition of $\mathcal{A}$ by $\underline{C^l}(x) = \left\lbrace a \in \mathcal{A} \mid \right. \left. \|a - x^l\| < \|a - x^j\|, \quad \forall j \neq l \right\rbrace$.
Moreover, denote $R_{lj}(t) = \min\limits_{a \in C^l(t)} \left\lbrace \|a - x^j(t)\| - \|a - x^l(t)\| \right\rbrace$ for all $1 \leq l,j \leq k$, and $r(t) = \min\limits_{l \neq j} R_{lj}$. \\
Due to Assumption \ref{StateEq23} we have that $\overline{C^l}(x(t)) = \underline{C^l}(x(t)) = C^l(t+1)$, for all $1 \leq l \leq k, \: t \in \mathbb{N}$, we also have that $r(t) > 0$ for all $t \in \mathbb{N}$.

\begin{proposition} \label{StateEq24}
Let $(C(t), x(t))$ be the clusters and centers k-means returns. Denote by $U = B\left( x^1(t),\frac{r(t)}{2}\right) \times  B\left( x^2(t),\frac{r(t)}{2}\right) \times \cdots \times B\left( x^l(t),\frac{r(t)}{2} \right)$ an open neighbourhood of $x(t)$, then for any $x \in U$ we have $C^l(t) = \underline{C^l}(x)$ for all $1 \leq l \leq k$.
\end{proposition}

\begin{proof}
Pick some $a \in C^l(t)$, then $x^l(t-1)$ is the closest center among the centers of $x(t-1)$. Since k-means halts at step $t$, then from (\ref{StateEq22}) we have $x(t)=x(t-1)$, thus $x^l(t)$ is the closest center to $a$ among the centers of $x(t)$. Further we have
\begin{equation}
	r(t) \leq \|x^j(t) - a\| - \|x^l(t) -a\| \quad \forall j \neq l . \label{StateEq25}
\end{equation}
Next, we show that $a \in \underline{C^l}(x)$, indeed
\begin{align*}
	\|a - x^l\| -  \|a - x^j\| &\leq \|a - x^l(t)\| + \|x^l(t) - x^l\| - \left( \|a - x^j(t)\| - \|x^j(t) - x^j\| \right) \\
	& = \|a - x^l\| - \|a - x^j(t)\| + \|x^l(t) - x^l\| + \|x^j(t) - x^j\| \\
	& < \|a - x^l\| - \|a - x^j(t)\| + r(t) \\
	& \leq -r(t) + r(t) = 0 ,
\end{align*}
where the second inequality holds since $x^l \in B\left( x^l(t), \frac{r(t)}{2} \right)$ and $x^j \in B\left( x^j(t), \frac{r(t)}{2} \right)$, and the third inequality follows from (\ref{StateEq25}), and we get that $C^l(t) \subseteq \underline{C^l}(x)$. 
By definition of $\underline{C^l}(x)$ we have that for any $l \neq j, \: \underline{C^l}(x) \cap \underline{C^j}(x)=\emptyset$, and for all $1 \leq l \leq k, \: \underline{C^l}(x) \subseteq \mathcal{A}$. Now, since $C(t)$ is a partition of $\mathcal{A}$, then $C^l(t) = \underline{C^l}(x)$ for all $1 \leq l \leq k$.
\end{proof}

\begin{proposition}[k-means converges to local minimum]
Let $(C(t), x(t))$ be the clusters and centers k-means returns, then $x(t)$ is local minimum of $F$ in $U = B\left( x^1(t),\frac{r(t)}{2}\right) \times  B\left( x^2(t),\frac{r(t)}{2}\right) \times \cdots \times B\left( x^l(t),\frac{r(t)}{2} \right) \subset \mathbb{R}^{nk}$.
\end{proposition}

\begin{proof}
The minimum of $F$ in $U$ is
\begin{equation*}
\min\limits_{x \in U} F(x) = \min\limits_{x \in U} \sum\limits_{l=1}^{k} \sum\limits_{a \in C^l(x)} \|a - x^l \|^2 = \min\limits_{x \in U} \sum\limits_{l=1}^{k} \sum\limits_{a \in C^l(t)} \|a - x^l \|^2 ,
\end{equation*}
where the last equality follows from \Cref{StateEq24}. \\
The function $x \mapsto \sum\limits_{l=1}^{k} \sum\limits_{a \in C^l(t)} \|a - x^l \|^2$ is strictly convex, separable in $x^l$ for all $1 \leq l \leq k$, and reaches its minimum at $\frac{1}{\left| C^l(t) \right|} \sum\limits_{a \in C^l(t)} a = mean(C^l(t)) = x^l(t),$ and the result follows.
\end{proof}
